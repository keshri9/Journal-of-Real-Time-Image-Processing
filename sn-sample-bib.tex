\begin{thebibliography}{9}
\bibitem{bib1} Zhu, Mengqiu, Lingjie Yu, Zongbiao Wang, Zhenxia Ke, and Chao Zhi. "A survey on objective evaluation of image sharpness." Applied Sciences 13, no. 4 (2023): 2652.
\end{thebibliography}


\bibitem{bib1} Shaojie Zhuo and Terence Sim. Defocus map estimation from a single image. Pattern Recognition, 44(9):1852–1858, 2011.

\bibitem{bib2} Pauline Trouv´e, Fr´ed´eric Champagnat, Guy Le Besnerais, Jacques
Sabater, Thierry Avignon, and J´erˆome Idier. Passive depth estimation
using chromatic aberration and a depth from defocus approach. Applied
optics, 52(29):7152–7164, 2013.

\bibitem{bib3} Anita Sellent and Paolo Favaro. Which side of the focal plane are
you on? In 2014 IEEE international conference on computational
photography (ICCP), pages 1–8. IEEE, 2014.
\bibitem{bib4} Mikko Kyt¨o, Mikko Nuutinen, and Pirkko Oittinen. Method for
measuring stereo camera depth accuracy based on stereoscopic vision.
In Three-Dimensional Imaging, Interaction, and Measurement, volume
7864, page 78640I. International Society for Optics and Photonics, 2011.

\bibitem{bib5} Rostam Affendi Hamzah and Haidi Ibrahim. Literature survey on stereo vision disparity map algorithms. Journal of Sensors, 2016, 2016.

\bibitem{bib6} Satyarth Praveen. Efficient depth estimation using sparse stereo-vision with other perception techniques. Coding Theory, page 111, 2020.

\bibitem{bib7}Yujiao Chen, Xiaorui Wang, Qiping Zhang,
Depth extraction method based on the regional feature points in integral imaging,
Optik,Volume 127, Issue 2,2016,Pages 763-768,ISSN 0030-4026,
https://doi.org/10.1016/j.ijleo.2015.10.171.
(https://www.sciencedirect.com/science/article/pii/S0030402615015399)
\bibitem{bib8} Camilo S´anchez-Ferreira, Jones Y Mori, Myl`ene CQ Farias, and Carlos H Llanos. A real-time stereo vision system for distance measurement
and underwater image restoration. Journal of the Brazilian Society of
Mechanical Sciences and Engineering, 38(7):2039–2049, 2016.

\bibitem{bib9} MNA Wahab, N Sivadev, and K Sundaraj. Development of monocular
vision system for depth estimation in mobile robot—robot soccer. In
Sustainable Utilization and Development in Engineering and Technology
(STUDENT), 2011 IEEE Conference on, pages 36–41. IEEE, 2011.

\bibitem{bib10} Chen Shan-shan, Zuo Wu-heng, and Feng Zhi-lin. Depth estimation via stereo vision using birchfield’s algorithm. In Communication Software
and Networks (ICCSN), 2011 IEEE 3rd International Conference on,
pages 403–407. IEEE, 2011.

\bibitem{bib11} Nuno Barroso Monteiro, Simao Marto, Joao Pedro Barreto, and Jos´e Gaspar. Depth range accuracy for plenoptic cameras. Computer Vision
and Image Understanding, 168:104–117, 2018.
\bibitem{bib12} Luca Palmieri, Gabriele Scrofani, Nicol`o Incardona, Genaro Saavedra, Manuel Mart´ınez-Corral, and Reinhard Koch. Robust depth estimation
for light field microscopy. Sensors, 19(3):500, 2019.

\bibitem{bib13} Alphonse, P.J.A., Sriharsha, K.V. Depth perception in single rgb camera system using lens aperture and object size: a geometrical approach for depth estimation. SN Appl. Sci. 3, 595 (2021). https://doi.org/10.1007/s42452-021-04212-4
\bibitem{bib14}Said Pertuz, Edith Pulido-Herrera, Joni-Kristian Kamarainen,
Focus model for metric depth estimation in standard plenoptic cameras,
ISPRS Journal of Photogrammetry and Remote Sensing,Volume 144,2018,Pages 38-47,ISSN 0924-2716,
https://doi.org/10.1016/j.isprsjprs.2018.06.020.
(https://www.sciencedirect.com/science/article/pii/S0924271618301837)
